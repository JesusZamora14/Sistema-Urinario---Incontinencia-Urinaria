
\documentclass[letterpaper,11pt]{article}
%%%%%%%%%%%%%%%%%%%%%%%%%%%%%%%%%%%%%%%%%%%%%%%%%%%%%%%%%%%%%%%%%%%%%%%%%%%%%%%%%%%%%%%%%%%%%%%%%%%%%%%%%%%%%%%%%%%%%%%%%%%%%%%%%%%%%%%%%%%%%%%%%%%%%%%%%%%%%%%%%%%%%%%%%%%%%%%%%%%%%%%%%%%%%%%%%%%%%%%%%%%%%%%%%%%%%%%%%%%%%%%%%%%%%%%%%%%%%%%%%%%%%%%%%%%%
\usepackage{graphicx}
\usepackage{amsmath,amsfonts,amssymb,amsthm,float}
\usepackage{hyperref}
\usepackage[utf8]{inputenc}
\usepackage[left=2cm, right=2cm, top=2cm, bottom=2cm]{geometry}

\setcounter{MaxMatrixCols}{10}
%TCIDATA{OutputFilter=LATEX.DLL}
%TCIDATA{Version=5.50.0.2953}
%TCIDATA{<META NAME="SaveForMode" CONTENT="1">}
%TCIDATA{BibliographyScheme=BibTeX}
%TCIDATA{LastRevised=Friday, December 13, 2024 17:43:26}
%TCIDATA{<META NAME="GraphicsSave" CONTENT="32">}
%TCIDATA{ComputeDefs=
%$R=10\times 10^{3}$
%$L=4.7\times 10^{-3}$
%$C=220\times 10^{-6}$
%}


\input{tcilatex}
\renewcommand{\baselinestretch}{1.15}
\setlength{\parindent}{0pt}
\setlength{\parskip}{0.5\baselineskip}
\pretolerance=2000 \tolerance=3000
\renewcommand{\abstractname}{Resumen}

\begin{document}

\title{Proyecto: Sistema Urinario}
\author{Marla Espinoza Bedoya (21212152); Abraham G\'{o}mez
Aguilar(21212158); \and Jesus Zamora Cervantes (21212185) \\
%EndAName
Departamento de Ingenier\'{\i}a El\'{e}ctrica y Electr\'{o}nica\\
Tecnol\'{o}gico Nacional de M\'{e}xico / Instituto Tecnol\'{o}gico de Tijuana%
}
\maketitle

\noindent \textbf{Palabras clave:} Uretra, Incontinencia, Filtraci\'{o}n
glomerular; Controlador; Orina.

\noindent Correo: \textbf{l21212152@tectijuana.edu.mx}

\noindent \noindent Carrera: \textbf{Ingenier\'{\i}a Biom\'{e}dica}

\noindent Asignatura: \textbf{Modelado de Sistemas Fisiol\'{o}gicos}

\noindent Profesor: \href{https://biomath.xyz/}{\textbf{Dr. Paul Antonio
Valle Trujillo}} (paul.valle@tectijuana.edu.mx)

\section{Funci\'{o}n de transferencia}

\subsection{Ecuaciones principales}

Se analiza el circuito utilizando la Ley de Voltajes de Kirchhoff y as\'{\i}
encontrar la ecuaci\'{o}n de la presi\'{o}n hidrost\'{a}tica sangu\'{\i}nea, 
$P_{h}(t)$ que llega a las nefronas en los ri\~{n}ones y la presi\'{o}n de
salida $P_{e}(t)$ que percibe la uretra y sus esf\'{\i}nteres que retienen
la orina, es decir, la entrada y salida del sistema respectivamente. El ri%
\~{n}\'{o}n realiza dos funciones principales: filtrar la sangre y
reabsorber el contenido, lo que da como consecuencia el desecho de lo que no
sirve, contenido en la orina. Para representar esto se utilizan dos
resistencias, la primera $R_{1}$ en la rama principal en serie con la presi%
\'{o}n de entrada cuyo flujo corresponde a la sangre que llega a los ri\~{n}%
ones $Fs(t)$, y la segunda en la primera rama secundaria, representando la
reabsorci\'{o}n, cuyo flujo es igual al flujo de sangre a los ri\~{n}ones
menos la orina generada como desecho, $Fs(t)-O(t).$ De tal manera que la
presi\'{o}n de entrada se representa como:

\begin{equation*}
P_{h}(t)=R_{1}Fs(t)+R_{2}\left[ Fs(t)-O(t)\right]
\end{equation*}

Luego en la segunda rama secundaria se representan el resto de los \'{o}%
rganos del sistema urinario, los ureteres al presentar movimientos perist%
\'{a}lticos, los cuales son contracciones que generan ondas que ayudan a
transportar la orina desde los ri\~{n}ones hacia la vejiga, impidiendo que
fluya en sentido contrario, se representa con un inductor $L$, cuyo flujo
corresponde a la orina $O(t)$. Luego llega a la vejiga, al cual est\'{a}
prevista de propiedades el\'{a}sticas que le permiten expandirse a medida
que se llena de orina, lo cual permite que sea representada con un capacitor 
$C$. Dicho esto la rama donde sucede la reabsorci\'{o}n y por done fluye la
orina puede ser igualada de la siguiente manera: 
\begin{equation*}
R_{2}\left[ Fs(t)-O(t)\right] =L\frac{dO(t)}{dt}+\frac{1}{C}\int
O(t)dt+R_{3}O(t)
\end{equation*}

Finalmente, cuando es momento de excretar la orina producida, se dirige a la
uretra la cual est\'{a} rodeada de esf\'{\i}nteres internos y externos, este 
\'{u}ltimo siendo voluntario, que permite liberar la orina de manera
consciente, se representa con una resistencia $R_{3},$cuyo flujo es el mismo
y por donde haya salida.%
\begin{equation*}
P_{e}(t)=R_{3}O(t)
\end{equation*}

\subsection{Transformada de Laplace}

Se aplica la transformada de Laplace a las ecuaciones principales, cuyo m%
\'{e}todo consiste en el intercambio de la variable dependiente por la letra
s, lo cual transforma las ecuaciones del dominio del tiempo al de
frecuencia. Se obtienen los siguientes resultados, para la funci\'{o}n de
entrada:%
\begin{equation*}
P_{h}(s)=R_{1}Fs(s)+R_{2}\left[ Fs(s)-O(s)\right] 
\end{equation*}

Para la segunda ecuaci\'{o}n principal se obtiene:%
\begin{equation*}
R_{2}\left[ Fs(s)-O(s)\right] =LsO(s)+\frac{O(s)}{Cs}+R_{3}O(s)
\end{equation*}

Para la ecuaci\'{o}n de salida del sistema se obtiene:%
\begin{equation*}
P_{e}(s)=R_{3}O(s)
\end{equation*}

\subsection{Procedimiento algebraico}

Para obtener la funci\'{o}n de transferencia se deben despejar las
ecuaciones transformadas a Laplace para que la entrada y salida est\'{e}n en
funci\'{o}n de una \'{u}nica variable dependiente, en este caso se busca
mantener ambas ecuaciones en funci\'{o}n de $O(s),$ ya que se puede observar
que la ecuaci\'{o}n de la presi\'{o}n de salida se encuentra \'{u}nicamente
en funci\'{o}n de la misma. Para esto se despeja el flujo de sangre
representado por $Fs(s)$. Se empieza por la segunda ecuaci\'{o}n principal,
despejando el \'{u}nico t\'{e}rmino que contiene $Fs(s)$:%
\begin{equation}
R_{2}Fs(s)-R_{2}O(s)=LsO(s)+\frac{O(s)}{Cs}+R_{3}O(s)
\end{equation}

\begin{equation*}
R_{2}Fs(s)=LsO(s)+\frac{O(s)}{Cs}+R_{3}O(s)+R_{2}O(s)
\end{equation*}

\begin{equation*}
R_{2}Fs(s)=\frac{LsO(s)Cs+O(s)+R_{3}O(s)Cs+R_{2}O(s)Cs}{Cs}
\end{equation*}

\begin{equation*}
Fs(s)=\frac{LsO(s)Cs+O(s)+R_{3}O(s)Cs+R_{2}O(s)Cs}{R_{2}Cs}
\end{equation*}

Una vez hecho esto, se factoriza la variable dependiente representada por $%
O(s)$, para ser suprimida con mayor facilidad posteriormente.%
\begin{equation*}
Fs(s)=O(s)\left[ \frac{LsCs+1+R_{3}Cs+R_{2}Cs}{R_{2}Cs}\right] 
\end{equation*}

Tomando la primera ecuaci\'{o}n principal, se sustituye la ahora despejada $%
Fs(s)$ en los t\'{e}rminos correspondientes:%
\begin{equation*}
P_{h}(s)=R_{1}Fs(s)+R_{2}Fs(s)-R_{2}O(s)
\end{equation*}%
\begin{equation*}
P_{h}(s)=Fs(s)\left[ R_{1}+R_{2}\right] -R_{2}O(s)
\end{equation*}%
\begin{equation*}
P_{h}(s)=O(s)\left[ \frac{LsCs+1+R_{3}Cs+R_{2}Cs}{R_{2}Cs}\right] \left[
R_{1}+R_{2}\right] -R_{2}O(s)
\end{equation*}

Se multiplica, se condensa todo para que todas las variables tengan el mismo
denominador y se reducen t\'{e}rminos. 
\begin{equation*}
P_{h}(s)=O(s)\left[ \frac{%
LCs^{2}R_{1}+LCs^{2}R_{2}+R_{1}+R_{2}+R_{3}CsR_{1}+R_{3}CsR_{2}+R_{2}CsR_{1}+\left( R_{2}\right) ^{2}Cs%
}{R_{2}Cs}\right] -R_{2}O(s)
\end{equation*}%
\begin{equation*}
P_{h}(s)=O(s)\left[ \frac{%
LCs^{2}R_{1}+LCs^{2}R_{2}+R_{1}+R_{2}+R_{3}CsR_{1}+R_{3}CsR_{2}+R_{2}CsR_{1}+\left( R_{2}\right) ^{2}Cs-\left( R_{2}\right) ^{2}Cs%
}{R_{2}Cs}\right] 
\end{equation*}

Se mantiene despejada el flujo $O(s)$ en la nueva ecuaci\'{o}n de entrada.

\begin{equation*}
P_{h}(s)=O(s)\left[ \frac{%
LCs^{2}R_{1}+LCs^{2}R_{2}+R_{1}+R_{2}+R_{3}CsR_{1}+R_{3}CsR_{2}+R_{2}CsR_{1}%
}{R_{2}Cs}\right] 
\end{equation*}%
Se aplica la funci\'{o}n de transferencia colocando a la entrada como el
denominador de la operaci\'{o}n y la salida como el numerador, en ambas se
factoriza el flujo $O(s)$ para eliminarlo de la ecuaci\'{o}n al ser un t\'{e}%
rminocontenido en ambas ecuaciones.

\begin{equation*}
\frac{P_{e}(s)}{P_{h}(s)}=\frac{R_{3}O(s)}{O(s)\left[ \frac{%
LCs^{2}R_{1}+LCs^{2}R_{2}+R_{1}+R_{2}+R_{3}CsR_{1}+R_{3}CsR_{2}+R_{2}CsR_{1}%
}{R_{2}Cs}\right] }
\end{equation*}%
\begin{equation*}
\frac{P_{e}(s)}{P_{h}(s)}=\frac{R_{3}}{\frac{%
LCs^{2}R_{1}+LCs^{2}R_{2}+R_{1}+R_{2}+R_{3}CsR_{1}+R_{3}CsR_{2}+R_{2}CsR_{1}%
}{R_{2}Cs}}
\end{equation*}

Se multiplica y se reducen t\'{e}rminos para obtener:%
\begin{equation*}
\frac{P_{e}(s)}{P_{h}(s)}=\frac{R_{2}R_{3}Cs}{%
LCs^{2}R_{1}+LCs^{2}R_{2}+R_{1}+R_{2}+R_{3}CsR_{1}+R_{3}CsR_{2}+R_{2}CsR_{1}}
\end{equation*}

Finalmente se agrupan las variables del mismo nivel y se cambia la prioridad
de los t\'{e}rminos, primero R, luego C y finalmente L.%
\begin{equation*}
\frac{P_{e}(s)}{P_{h}(s)}=\frac{R_{2}R_{3}Cs}{\left( R_{1}CL+R_{2}CL\right)
s^{2}+\left( R_{1}R_{3}C+R_{2}R_{3}C+R_{1}R_{2}C\right) s+R_{1}+R_{2}}
\end{equation*}

\subsection{Resultado}

La funci\'{o}n de transferencia obtenida para la representaci\'{o}n del
sistema urinario es:%
\begin{equation*}
\frac{P_{e}(s)}{P_{h}(s)}=\frac{R_{2}R_{3}Cs}{\left( R_{1}CL+R_{2}CL\right)
s^{2}+\left( R_{1}R_{3}C+R_{2}R_{3}C+R_{1}R_{2}C\right) s+R_{1}+R_{2}}
\end{equation*}

\section{Estabilidad del sistema en lazo abierto}

Para determinar la estabilidad del sistema en lazo abierto, se calculan los
polos de la funci\'{o}n de transferencia, es decir:%
\begin{equation*}
a_{2}s^{2}+a_{1}s+a_{0}=0
\end{equation*}

donde las variables son:%
\begin{eqnarray*}
a_{2} &=&R_{1}CL+R_{2}CL \\
a_{1} &=&R_{1}R_{3}C+R_{2}R_{3}C+R_{1}R_{2}C \\
a_{0} &=&R_{1}+R_{2}
\end{eqnarray*}%
Valores de par\'{a}metros para el control y el caso:%
\begin{equation*}
R_{1}=10\times 10^{3}
\end{equation*}%
\begin{equation*}
R_{2}=1000\times 10^{3}
\end{equation*}%
\begin{equation*}
L=4.7\times 10^{-3}
\end{equation*}

Estabilidad para el control (individuo saludable):%
\begin{equation*}
C=440\times 10^{-6}
\end{equation*}%
\begin{equation*}
R_{3}=20\times 10^{3}
\end{equation*}%
\begin{equation*}
2.\,\allowbreak 088\,7s^{2}+1.\,\allowbreak 3288\times 10^{7}s+0.6=0.0
\end{equation*}

En donde las ra\'{\i}ces de la ecuaci\'{o}n de segundo orden son:%
\begin{equation*}
\lambda _{1}=-4.\,\allowbreak 515\,4\times 10^{-8}
\end{equation*}%
\begin{equation*}
\lambda _{2}=-6.\,\allowbreak 361\,9\times 10^{6}
\end{equation*}

Para el caso del individuo enfermo (paciente con incontinencia):%
\begin{equation*}
C=110\times 10^{-6}
\end{equation*}%
\begin{equation*}
R_{3}=5\times 10^{3}
\end{equation*}%
\begin{equation*}
0.522\,17s^{2}+1655.5\times 10^{3}s+0.6=0.0
\end{equation*}

Las ra\'{\i}ces obtenidas son las siguientes:%
\begin{equation*}
\lambda _{2}=-3.\,\allowbreak 624\,3\times 10^{-7}
\end{equation*}%
\begin{equation*}
\lambda _{2}=-3.\,\allowbreak 624\,3\times 10^{-7}
\end{equation*}

Se observa que las ambos pares de ra\'{\i}ces para control y caso son reales
y negativas, por lo tanto, se concluye que ambos sistemas son estables.

\section{Modelo de ecuaciones integro-diferenciales}

Para obtener el modelo de ecuaciones integro-diferenciales se despejan los
flujos o corrientes, en este caso correspondientes al flujo de sangre de
entrada y la orina, junto con la salida, siendo la presi\'{o}n en la uretra,
ya que corresponden a las variables dependientes del sistema. Se hace a
partir de las ecuaciones principales:%
\begin{eqnarray*}
Fs(t) &=&\left[ P_{h}(t)-R_{1}Fs(t)+R_{2}O(t)\right] \left[ \frac{1}{R_{2}}%
\right] , \\
O(t) &=&\left[ R_{2}Fs(t)-L\frac{dO(t)}{dt}-\frac{1}{C}\int O(t)dt-R_{3}O(t)%
\right] \left[ \frac{1}{R_{2}}\right]  \\
P_{e}(t) &=&R_{3}O(t)
\end{eqnarray*}

\section{Error en estado estacionario}

Para calcular el error en estado estacionario, es necesario calcular el
siguiente l\'{\i}mite, donde R(s) representa un escal\'{o}n de entrada [1/s].%
\begin{equation*}
e\left( t\right) =\lim_{s\rightarrow 0}sR\left( s\right) \left[ 1-\dfrac{%
P_{h}\left( s\right) }{P_{e}\left( s\right) }\right] =\lim_{s\rightarrow 0}s%
\frac{1}{s}\left[ 1-\frac{R_{2}R_{3}Cs}{\left( R_{1}CL+R_{2}CL\right)
s^{2}+\left( R_{1}R_{3}C+R_{2}R_{3}C+R_{1}R_{2}C\right) s+R_{1}+R_{2}}\right]
\end{equation*}%
\begin{equation*}
=1V.
\end{equation*}%
\begin{equation*}
R(s):\text{Representa la entrada al sistema [el escal\'{o}n }\frac{1}{s}%
\text{]}
\end{equation*}%
\begin{equation*}
\dfrac{P_{h}\left( s\right) }{P_{e}\left( s\right) }:\text{Representa la
entrada al sistema [el escal\'{o}n }\frac{1}{s}\text{]}
\end{equation*}

\section{C\'{a}lculo de componentes para el controlador PID}

Para obtener los valores de ganancia que permitan elaborar un controlador
PID se grafica la respuesta de presi\'{o}n del paciente sano (control), el
tratamiento y el caso de paciente enfermo del sistema urinario en lazo
cerrado configurando la velocidad de la respuesta para que el sobreimpulso
no sea mayor a 10\% (esto porque se busca mantener la respuesta del
tratamiento sea lo m\'{a}s parecida a la se\~{n}al de control para que sea
efectivo el controlador), al realizar esto se obtiene un controlador de tipo
I, es decir solo se tiene la parte integral del mismo, lo quiere decir que
es \'{u}til para eliminar el error en estado estacionario a la salida del
sistema. El valor de ganancia obtenido para dicho elemento de control tiene
un valor de:%
\begin{equation*}
k_{I}=\frac{1}{R_{e}C_{r}}=1317.4583
\end{equation*}

\begin{equation*}
C_{r}=1\times 10^{-6}
\end{equation*}

Se propone un valor de capacitor de Cr, para obtener los valores de los
componentes electr\'{o}nicos necesarios para que el controlador regule
adecuadamente el sistema, que es el siguiente:

\begin{equation*}
R_{e}=\frac{1}{k_{I}C_{r}}=\frac{1}{(1317.4583)(1\times 10^{-6})}%
=759.04\Omega 
\end{equation*}

\end{document}
